\addchap{List of abbreviations} 


Throughout this thesis, phonetic and phonemic transcriptions are stated according to the \citet{IPA1999}. Phonetic acoustic transcriptions are enclosed in square brackets []. Phonological representations according to a particular analysis are indicated by slashes //. In addition to a phonemic transcription, these representations can also entail information about syllabicity and syllable boundaries (see below). In line with common practice in academic publications on Berber languages, geminate consonants are transcribed as a double consonant (e.g. /tt/).


\bigskip 
% \begin{table}
  \begin{tabular}{ll}
%     \lsptoprule
%     Abbreviation & Meaning \\
%     \midrule
-	 & word-internal morpheme boundary or small phrase edge \\
 	 & tone \\
* 	 & (asterisk) precedes an ungrammatical form or indicates a  \\ 
	 & starred tone \\ 
.	 & (period) syllable boundary \\
+	 & (plus) indicates the juncture of two tonal targets belonging   \\ 
	 & to the same tonal complex \\
\%	 & (percent) indicates an intonation phrase edge tone \\
s̩ g̍	 & (subscript/superscript vertical line) marks the syllable   \\
	 & nucleus according to a particular analysis \\
AD	 & /ad/ complementiser \\
AM  & 	Autosegmental-Metrical model \\
BS	 & bound state \\
C	 & consonant or segment that does not occupy the syllable  \\
 	 & nucleus \\
CS 	 & contrastive statement \\
dB	 & decibel  \\
EQ	 & echo question \\
EEQT	 & Eastern European question tune \\
% \lspbottomrule
  \end{tabular}
%   \caption{Abbreviations used throughout this book}
% \end{table} 

% \begin{table}
  \begin{tabular}{ll}
%     \lsptoprule
%     Abbreviation & Meaning \\
%     \midrule
F	 & feminine or final syllable \\
F0	 & fundamental frequency \\
H	 & high tonal target \\
Hz	 & hertz = frequency of cycles per second \\
INT	 & /is/ interrogative preverb \\
IPO	 & Institute for Perception Research in Eindhoven \\ 
L 	 & liquid or low tonal target \\
LRT	 & Likelihood Ratio Test \\
M 	 & masculine \\
N	 & nasal \\
OT	 & Optimality Theory \\
PL	 & plural \\
PTEV	 & postlexically triggered epenthetic vowel \\
PU	 & penultimate syllable \\
S	 & sonorant consonant \\
SE	 & standard error \\
SG	 & singular \\
ST	 & semitones \\
TBU	 & tone bearing unit \\
V	 & vowel or segment that occupies syllable nucleus \\
VOT	 & voice onset time \\
Y/N  & 	yes-no question \\
β	 & estimated coefficient \\
% \lspbottomrule
  \end{tabular}
%   \caption{Abbreviations used throughout this book (continued)}
% \end{table} 
